\documentclass[letterpaper,11pt]{article}
\usepackage{setspace} 
% \usepackage{filecontents}
% \begin{filecontents}{myrefs.bib}
% @article{pred,
%   author    = {Davi Pereira-Santos and Gabriel Dalforno and Andre Carlos P L F Carvalho},
%   title     = {Predictable universally unique identification of sequential events on complex objects},
%   journal   = {Soon on arXiv / Elsevier submission},
%   year      = {2021},
% }
% \end{filecontents}


% Choose bibliography style for formatting list of publications
\usepackage[style=ieee,url=false,doi=false,maxbibnames=99,sorting=ydnt,dashed=false]{biblatex}
% \addbibresource{myrefs.bib}
\bibliography{papers}

% Choose theme, e.g. black, RedViolet, ForestGreen, MidnightBlue
\def\theme{MidnightBlue}

% More predefined colors can be found in 
% https://en.wikibooks.org/wiki/LaTeX/Colors
% Example photograph taken from Wikimedia Commons
% https://commons.wikimedia.org/wiki/File:Kiara_Krit_passport.jpg

\usepackage{simplecv}
\boldname{Pereira dos Santos}{Davi}{N.}
\usepackage{ragged2e}

\begin{document}
\headinginline{Letter of Interest}{}
\vspace{40pt}

{\justifying
\Large
\setstretch{1.0}
Dear Recruiter,
\vspace{10pt}

I found out about Microsoft Research positions through the work of Simon Peyton Jones on Functional Programming. 
I would like to perform research on this subject and I hope the Calc Intelligence position is a good oportunity for that. 

Often I find myself imagining, and sometimes trying, ways to make programming languages and even spreadsheets more intelligent and less error-prone.
I believe data and process will converge to a single concept - as shown in my last submitted work \cite{pereirasantos2021predictable}, where I focused only on the identification part.

I have a growing research track in Machine Learning along with participation in projects as a software architect/developer, from design to deploy.
One of them was a prototypic platform for AutoML.

}

{\justifying
\Large
\setstretch{1.0}
% Modeling parts of the world into computational structures and abstract thinking have been one of my best qualities, as can be seen in my last submitted paper and software.
Most of my career I spent learning, and hope to keep learning and producing knowledge at Microsoft.
I feel motivated at creative environments and one of my driving forces is the belief to be contributing to society.
\vspace{20pt}

Please, do not hesitate to contact any of my direct coleagues or supervisors:

Prof. Adriano Rivolli - rivolli@utfpr.edu.br

Prof. Luís Paulo F. Garcia - luis.garcia@unb.br

Dr. Rafael Bizão - rabizao@gmail.com

Prof. João Batista - jbatista@icmc.usp.br

Prof. Joao Pedro Pedroso - jpp@fc.up.pt

Prof. André Carvalho - andre@icmc.usp.br

}

{\justifying
\Large
\setstretch{1.0}
\vspace{30pt}
\hspace{420pt} Sincerely yours,
\vspace{10pt}

\hspace{420pt} Davi
}

\newpage
% Heading
\headinginline{Davi Pereira dos Santos}{
    % Website: \website{example.com} \\ 
    \email{dpsabc@gmail.com} \\
    % LinkedIn: \linkedin{name-surname} \\
    \github{davips}
}

% \headingphoto{Name Surname}{
%     Website: \website{example.com} \\ 
%     Email: \email{example@example.edu} \\
%     LinkedIn: \linkedin{name-surname} \\
%     GitHub: \github{example}
% }{photo.jpg}

\vspace{0.35cm}
{\justifying
I am a Brazilian/Polish citizen looking for a research position or a position that bridges software development and scientific applications. 
In the long term, I am particularly interested in software ergonomics (type systems, functional programming, syntax, everyday tools for programmers and also non-programmers).
Currently, I am working on applications\footnote{https://pypi.org/project/garoupa and https://pypi.org/project/ldict} in data science and programming in general that take advantage of my proposed algebra-based identification concept \cite{pereirasantos2021predictable}. 
My career intertwins academic formation, periods in industry and professional leave due to personal reasons.
Therefore, the chronology presented here might seem unusual depending on the reader perspective and background.

}

% Page One
\import{sections/}{education.tex}
\import{sections/}{experience.tex}
\pagebreak
\sidebyside
    {\import{sections/}{skills.tex}}
    {\import{sections/}{languages.tex}}


% Page Two
\import{sections/}{teaching.tex}
\import{sections/}{projects.tex}
\import{sections/}{extracurricular.tex}
\import{sections/}{awards.tex}
\import{sections/}{publications.tex}

\end{document}
