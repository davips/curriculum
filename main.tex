\documentclass[letterpaper,11pt]{article}
\usepackage{setspace} 
% \usepackage{filecontents}
% \begin{filecontents}{myrefs.bib}
% @article{pred,
%   author    = {Davi Pereira-Santos and Gabriel Dalforno and Andre Carlos P L F Carvalho},
%   title     = {Predictable universally unique identification of sequential events on complex objects},
%   journal   = {Soon on arXiv / Elsevier submission},
%   year      = {2021},
% }
% \end{filecontents}


% Choose bibliography style for formatting list of publications
\usepackage[style=ieee,url=false,doi=false,maxbibnames=99,sorting=ydnt,dashed=false]{biblatex}
% \addbibresource{myrefs.bib}
\bibliography{papers}

% Choose theme, e.g. black, RedViolet, ForestGreen, MidnightBlue
\def\theme{MidnightBlue}

% More predefined colors can be found in 
% https://en.wikibooks.org/wiki/LaTeX/Colors
% Example photograph taken from Wikimedia Commons
% https://commons.wikimedia.org/wiki/File:Kiara_Krit_passport.jpg

\usepackage{simplecv}
\boldname{Pereira dos Santos}{Davi}{N.}
\usepackage{ragged2e}

\begin{document}
% \headinginline{Letter of Interest}{}
% \vspace{40pt}
% 
% {\justifying
% \Large
% \setstretch{1.0}
% Dear Recruiter,
% \vspace{10pt}
% 
% I found out about Microsoft Research positions through the work of Simon Peyton Jones on Functional Programming. 
% I would like to perform research on this subject and I hope the Calc Intelligence position is a good oportunity for that. 
% 
% Often I find myself imagining, and sometimes trying, ways to make programming languages and even spreadsheets more intelligent and less error-prone.
% I believe data and process will converge to a single concept - as shown in my last submitted work \cite{pereirasantos2021predictable}, where I focused only on the identification part.
% 
% I am looking for a machine learning/data analytics/research position where I can see my knowledge and previous experience helping to solve problems which require at least some level of creativity and where I can learn interesting subjects while also providing guidance to others. %
% 
% The \textit{Model Risk \& Review} position at Nubank seems to fit these features well along with the values of the company which I also share.
% 
% I have a previous professional and academic experience, and a growing research and development track in Computer Science as a software architect/developer, from design to deploy.
% This included a prototypic end-to-end platform for Machine Learning and an on-line repository for scientific datasets.
% 
% Recently I have implemented an innovative universal identification system for data and processes, based on functional programming and abstract algebra, to simplify caching distributed machine learning pipelines among  many other possible applications.
% 
% In the long term, I am particularly interested in software/business process/interface ergonomics, resorting to type systems, functional programming and human-friendly syntax as everyday tools for programmers and also non-programmers.
% 
% }
% 
% {\justifying
% \Large
% \setstretch{1.0}
% Modeling parts of the world into computational structures and abstract thinking have been one of my best qualities, as can be seen in my last submitted paper and software.
% 
% Most of my career I spent learning, and hope to continue learning.
% I feel motivated at creative environments and one of my driving forces is the belief to be contributing to society.
% 
% \vspace{20pt}
% 
% Please, do not hesitate to contact any of my direct coleagues or supervisors:
% 
% Prof. Adriano Rivolli - rivolli@utfpr.edu.br
% 
% Prof. Luís Paulo F. Garcia - luis.garcia@unb.br
% 
% Dr. Rafael Bizão - rabizao@gmail.com
% 
% Prof. João Batista - jbatista@icmc.usp.br
% 
% Prof. Joao Pedro Pedroso - jpp@fc.up.pt
% 
% Prof. André Carvalho - andre@icmc.usp.br
% 
% }
% 
% {\justifying
% \Large
% \setstretch{1.0}
% \vspace{30pt}
% \hspace{420pt} Sincerely yours,
% \vspace{10pt}
% 
% \hspace{420pt} Davi
% }








\newpage
% Heading
\headinginline{Davi Pereira dos Santos}{
    % Website: \website{example.com} \\ 
    \email{dpsabc@gmail.com} \\
    % LinkedIn: \linkedin{name-surname} \\
    \github{davips}
}

% \headingphoto{Name Surname}{
%     Website: \website{example.com} \\ 
%     Email: \email{example@example.edu} \\
%     LinkedIn: \linkedin{name-surname} \\
%     GitHub: \github{example}
% }{photo.jpg}

\vspace{0.35cm}
{\justifying
I am a Brazilian-Polish researcher in Computer Science currently applied to health sciences. %
I am also interested in programming language design and tools. %

% , therefore able to stay and work long periods in the Schengen area without visa delays, if needed.
% Currently, I am working on applications\footnote{https://pypi.org/project/garoupa and https://pypi.org/project/ldict} in data science and programming in general \cite{pereirasantos2021predictable}. 
% My career intertwins academic formation, teaching, periods in industry and professional leave due to personal reasons.
% Therefore, the chronology presented here might seem unusual depending on the reader perspective and background.

}
% I am a Brazilian/Polish citizen 
% looking for a developer and/or research position. 
% I am particularly 
% with a solid background in Machine Learning interested in programming ergonomics (type systems, functional programming, automatic identifiers, intuitive syntax) applied to fields like Data Science. 
% or any field where the number of users from areas not necessarily related to Computer Science is increasing, or where even experienced developers are willing to think about algorithm implementation in a new perspective.
% I am also interested in providing innovative tools to ease the programming task (IDE, version control, language/library design, etc.) in general or for a specific product development.

% Page One
\import{sections/}{education.tex}
\import{sections/}{experience.tex}
\newpage
\sidebyside
    {\import{sections/}{skills.tex}}
    {\import{sections/}{languages.tex}}


% Page Two
\import{sections/}{projects.tex}
\import{sections/}{extracurricular.tex}
\import{sections/}{teaching.tex}

\import{sections/}{awards.tex}
\import{sections/}{publications.tex}

\end{document}
